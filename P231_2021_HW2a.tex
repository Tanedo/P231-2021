\documentclass[12pt]{article}
%% arXiv paper template by Flip Tanedo
%% last updated: Dec 2016


%%%%%%%%%%%%%%%%%%%%%%%%%%%%%
%%%  THE USUAL PACKAGES  %%%%
%%%%%%%%%%%%%%%%%%%%%%%%%%%%%

\usepackage{amsmath}
\usepackage{amssymb}
\usepackage{amsfonts}
\usepackage{graphicx}
\usepackage{xcolor}
\usepackage{nopageno}
\usepackage{enumerate}
\usepackage{parskip}

%%%%%%%%%%%%%%%%%%%%%%%%%%%%%%%%%
%%%  UNUSUAL PACKAGES        %%%%
%%%  Uncomment as necessary. %%%%
%%%%%%%%%%%%%%%%%%%%%%%%%%%%%%%%%

%% MATH AND PHYSICS SYMBOLS
%% ------------------------
%\usepackage{slashed}       % \slashed{k}
%\usepackage{mathrsfs}      % Weinberg-esque letters
%\usepackage{youngtab}	    % Young Tableaux
%\usepackage{pifont}        % check marks
\usepackage{bbm}           % \mathbbm{1} incomp. w/ XeLaTeX 
%\usepackage[normalem]{ulem} % for \sout


%% CONTENT FORMAT AND DESIGN (below for general formatting)
%% --------------------------------------------------------
\usepackage{lipsum}        % block of text (formatting test)
%\usepackage{color}         % \color{...}, colored text
%\usepackage{framed}        % boxed remarks
%\usepackage{subcaption}    % subfigures; subfig depreciated
%\usepackage{paralist}      % compactitem
%\usepackage{appendix}      % subappendices
%\usepackage{cite}          % group cites (conflict: collref)
%\usepackage{tocloft}       % Table of Contents	

%% TABLES IN LaTeX
%% ---------------
%\usepackage{booktabs}      % professional tables
%\usepackage{nicefrac}      % fractions in tables,
%\usepackage{multirow}      % multirow elements in a table
%\usepackage{arydshln} 	    % dashed lines in arrays

%% Other Packages and Notes
%% ------------------------
%\usepackage[font=small]{caption} % caption font is small



\renewcommand{\thesection}{}
\renewcommand{\thesubsection}{\arabic{subsection}}

%%%%%%%%%%%%%%%%%%%%%%%%%%%%%%%%%%%%%%%%%%%%%%%
%%%  PAGE FORMATTING and (RE)NEW COMMANDS  %%%%
%%%%%%%%%%%%%%%%%%%%%%%%%%%%%%%%%%%%%%%%%%%%%%%

\usepackage[margin=2cm]{geometry}   % reasonable margins

\graphicspath{{figures/}}	        % set directory for figures

% for capitalized things
\newcommand{\acro}[1]{\textsc{\MakeLowercase{#1}}}    

\numberwithin{equation}{subsection}    % set equation numbering
\renewcommand{\tilde}{\widetilde}   % tilde over characters
\renewcommand{\vec}[1]{\mathbf{#1}} % vectors are boldface

\newcommand{\dbar}{d\mkern-6mu\mathchar'26}    % for d/2pi
\newcommand{\ket}[1]{\left|#1\right\rangle}    % <#1|
\newcommand{\bra}[1]{\left\langle#1\right|}    % |#1>
\newcommand{\Xmark}{\text{\sffamily X}}        % cross out


\let\olditemize\itemize
\renewcommand{\itemize}{
  \olditemize
  \setlength{\itemsep}{1pt}
  \setlength{\parskip}{0pt}
  \setlength{\parsep}{0pt}
}


% Commands for temporary comments
\newcommand{\comment}[2]{\textcolor{red}{[\textbf{#1} #2]}}
\newcommand{\flip}[1]{{\color{red} [\textbf{Flip}: {#1}]}}
\newcommand{\email}[1]{\texttt{\href{mailto:#1}{#1}}}

\newenvironment{institutions}[1][2em]{\begin{list}{}{\setlength\leftmargin{#1}\setlength\rightmargin{#1}}\item[]}{\end{list}}


\usepackage{fancyhdr}		% to put preprint number



%%%%%%%%%%%%%%%%%%%
%%%  HYPERREF  %%%%
%%%%%%%%%%%%%%%%%%%

%% This package has to be at the end; can lead to conflicts
\usepackage{microtype}
\usepackage[
	colorlinks=true,
	citecolor=black,
	linkcolor=black,
	urlcolor=green!50!black,
	hypertexnames=false]{hyperref}



%%%%%%%%%%%%%%%%%%%%%
%%%  TITLE DATA  %%%%
%%%%%%%%%%%%%%%%%%%%%

\begin{document}


\begin{center}

    {\Large \textsc{Homework 2a:} 
    \textbf{Self-Adjointness}}
    
\end{center}

\vskip .4cm

\noindent
\begin{tabular*}{\textwidth}{rlcrll}
	\textsc{Course:}& Physics 231, \emph{Methods of Theoretical Physics} (2021)
	&
%	\hspace{1.2cm}
	&
	\\
	\textsc{Instructor:}& Professor Flip Tanedo (\email{flip.tanedo@ucr.edu})
	&
	%\hfill
	&
	& 
	\\
	\textsc{Due by:}& Wed, October 13
	&
	%\hfill
	&
	%	
\end{tabular*}

\subsection{The $i$ in the momentum operator} 

In one-dimensional quantum mechanics the momentum operator is defined to be
\begin{align}
	\hat p = -i \frac{d}{dx} \ ,
\end{align}
where we set $\hbar = 1$.
Suppose the vector space of wavefunctions is composed of complex functions $\psi(x)$ over the real line that are square integrable\footnote{This basically means that $\psi(x)\to 0$ sufficiently fast for $x\to \pm \infty$. This means that you can integrate by parts without worrying about boundary terms.}. The inner product on this space is
\begin{align}
	\langle \psi, \chi \rangle = \int_{-\infty}^\infty dx\; \psi^*(x)\, \chi(x) \ .
\end{align}
Recall that an operator $\mathcal O$ is self-adjoint (Hermitian) if $\mathcal O^\dag = \mathcal O$. This is defined by:
\begin{align}
	\langle \mathcal O\psi, \chi \rangle
	= 
	\langle \psi,  \mathcal O \chi \rangle \ .
\end{align} 
Show that $\hat p$ is self-adjoint. Comment on why it is important that $\hat p$ is defined to have a factor of $i$ in it. Comment on whether or not the overall sign is meaningful.


\section{Extra Credit}

These problems are not graded and are for your edification. You are strongly encouraged to explore and discuss these topics, especially if they are in a field of interest to you.


\subsection{A two-dimensional function space} 

%https://math.stackexchange.com/questions/942263/really-advanced-techniques-of-integration-definite-or-indefinite/943212

Let us ignore the subtleties of defining a function space (metric, domain, boundary conditions). Instead, let's construct a cute two-dimensional function space that gives us a shortcut to calculate a particular \emph{indefinite} integral. 

Consider a two dimensional vector space spanned by the functions
\begin{align}
	\left|f_1\right\rangle
	&= f_1(x) = 
	e^{ax} \cos bx
	&
	\left|f_2\right\rangle
	&=
	f_2(x) = 
	e^{ax} \sin bx \ ,
\end{align}
where $a$ and $b$ are constants. Forget orthonormality or boundary conditions for this problem. The derivative $d/dx$ is a linear operator that acts on this space. Write down the derivative as a $2\times 2$ matrix in the above basis, $D$.

Invert $D$ in the usual way that you learned to invert $2\times 2$ matrices during your childhood\footnote{Stuck? Here's a life pro tip: \url{http://bfy.tw/KG2Z}}. Call this matrix $D^{-1}$. 

Now stop and think: the inverse of a derivative is an indefinite integral\footnote{Ignore the constant term.}. Thus acting with $D^{-1}$ on the vector $|f_1\rangle$ should be understood as an integral of $f_1(x)$. Show that, indeed,
\begin{align}
	D^{-1} |f_1\rangle = \int dx\, e^{ax} \cos bx \ .
\end{align}
Feel free to use \emph{Mathematica} to do the indefinite integral on the right-hand side. Pat yourself on the back if you can do it without a computer.




\end{document}