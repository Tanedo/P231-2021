\documentclass[12pt]{article}
%% arXiv paper template by Flip Tanedo
%% last updated: Dec 2016


%%%%%%%%%%%%%%%%%%%%%%%%%%%%%
%%%  THE USUAL PACKAGES  %%%%
%%%%%%%%%%%%%%%%%%%%%%%%%%%%%

\usepackage{amsmath}
\usepackage{amssymb}
\usepackage{amsfonts}
\usepackage{graphicx}
\usepackage{xcolor}
\usepackage{nopageno}
\usepackage{enumerate}
\usepackage{parskip}

%%%%%%%%%%%%%%%%%%%%%%%%%%%%%%%%%
%%%  UNUSUAL PACKAGES        %%%%
%%%  Uncomment as necessary. %%%%
%%%%%%%%%%%%%%%%%%%%%%%%%%%%%%%%%

%% MATH AND PHYSICS SYMBOLS
%% ------------------------
%\usepackage{slashed}       % \slashed{k}
%\usepackage{mathrsfs}      % Weinberg-esque letters
%\usepackage{youngtab}	    % Young Tableaux
%\usepackage{pifont}        % check marks
\usepackage{bbm}           % \mathbbm{1} incomp. w/ XeLaTeX 
%\usepackage[normalem]{ulem} % for \sout


%% CONTENT FORMAT AND DESIGN (below for general formatting)
%% --------------------------------------------------------
\usepackage{lipsum}        % block of text (formatting test)
%\usepackage{color}         % \color{...}, colored text
%\usepackage{framed}        % boxed remarks
%\usepackage{subcaption}    % subfigures; subfig depreciated
%\usepackage{paralist}      % compactitem
%\usepackage{appendix}      % subappendices
%\usepackage{cite}          % group cites (conflict: collref)
%\usepackage{tocloft}       % Table of Contents	

%% TABLES IN LaTeX
%% ---------------
%\usepackage{booktabs}      % professional tables
%\usepackage{nicefrac}      % fractions in tables,
%\usepackage{multirow}      % multirow elements in a table
%\usepackage{arydshln} 	    % dashed lines in arrays

%% Other Packages and Notes
%% ------------------------
%\usepackage[font=small]{caption} % caption font is small



\renewcommand{\thesection}{}
\renewcommand{\thesubsection}{\arabic{subsection}}

%%%%%%%%%%%%%%%%%%%%%%%%%%%%%%%%%%%%%%%%%%%%%%%
%%%  PAGE FORMATTING and (RE)NEW COMMANDS  %%%%
%%%%%%%%%%%%%%%%%%%%%%%%%%%%%%%%%%%%%%%%%%%%%%%

\usepackage[margin=2cm]{geometry}   % reasonable margins

\graphicspath{{figures/}}	        % set directory for figures

% for capitalized things
\newcommand{\acro}[1]{\textsc{\MakeLowercase{#1}}}    

\numberwithin{equation}{subsection}    % set equation numbering
\renewcommand{\tilde}{\widetilde}   % tilde over characters
\renewcommand{\vec}[1]{\mathbf{#1}} % vectors are boldface

\newcommand{\dbar}{d\mkern-6mu\mathchar'26}    % for d/2pi
\newcommand{\ket}[1]{\left|#1\right\rangle}    % <#1|
\newcommand{\bra}[1]{\left\langle#1\right|}    % |#1>
\newcommand{\Xmark}{\text{\sffamily X}}        % cross out


\let\olditemize\itemize
\renewcommand{\itemize}{
  \olditemize
  \setlength{\itemsep}{1pt}
  \setlength{\parskip}{0pt}
  \setlength{\parsep}{0pt}
}


% Commands for temporary comments
\newcommand{\comment}[2]{\textcolor{red}{[\textbf{#1} #2]}}
\newcommand{\flip}[1]{{\color{red} [\textbf{Flip}: {#1}]}}
\newcommand{\email}[1]{\texttt{\href{mailto:#1}{#1}}}

\newenvironment{institutions}[1][2em]{\begin{list}{}{\setlength\leftmargin{#1}\setlength\rightmargin{#1}}\item[]}{\end{list}}


\usepackage{fancyhdr}		% to put preprint number



%%%%%%%%%%%%%%%%%%%
%%%  HYPERREF  %%%%
%%%%%%%%%%%%%%%%%%%

%% This package has to be at the end; can lead to conflicts
\usepackage{microtype}
\usepackage[
	colorlinks=true,
	citecolor=black,
	linkcolor=black,
	urlcolor=green!50!black,
	hypertexnames=false]{hyperref}



%%%%%%%%%%%%%%%%%%%%%
%%%  TITLE DATA  %%%%
%%%%%%%%%%%%%%%%%%%%%

\begin{document}


\begin{center}

    {\Large \textsc{Homework 3a:} 
    \textbf{Using the Residue Theorem}}
    
\end{center}

\vskip .4cm

\noindent
\begin{tabular*}{\textwidth}{rlcrll}
	\textsc{Course:}& Physics 231, \emph{Methods of Theoretical Physics} (2021)
	&
%	\hspace{1.2cm}
	&
	\\
	\textsc{Instructor:}& Professor Flip Tanedo (\email{flip.tanedo@ucr.edu})
	&
	%\hfill
	&
	& 
	\\
	\textsc{Due by:}& Wed, Nov 3
	&
	%\hfill
	&
	%	
\end{tabular*}

\subsection{Which contour?}

Suppose you would like to calculate the integral
\begin{align}
 \int_{-\infty}^\infty	dx \, \frac{e^{iax}}{x^4+1} \ ,
\end{align}
where $a$ is either $+1$ or $-1$.
\begin{enumerate}
\item Identify the poles of the integrand and the residue of each pole.
\item There are two natural contours you can use to compute the integral using the residue theorem. Draw the contours: one is in the upper half-plane, the other is in the lower-half plane. Note whether the have positive or negative orientation.
\item When $a=+1$, which contour has an integral that converges (goes to zero) on the arc? Which poles are enclosed and contribute their residues to the integral?
\item When $a=-1$, which contour has an integral that converges (goes to zero) on the arc? Which poles are enclosed and contribute their residues to the integral?
\end{enumerate}
You do not have to perform the integral, but if you feel like your life would not be complete without doing so, you can feel free to do so for $a= \pm 1$.
  



\end{document}