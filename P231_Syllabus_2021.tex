\documentclass[12pt]{article}


%%%%%%%%%%%%%%%%%%%%%%%%%%%%%
%%%  THE USUAL PACKAGES  %%%%
%%%%%%%%%%%%%%%%%%%%%%%%%%%%%

\usepackage{amsmath}
\usepackage{amssymb}
\usepackage{amsfonts}
\usepackage{graphicx}
\usepackage{xcolor}
\usepackage{nopageno}

%%%%%%%%%%%%%%%%%%%%%%%%%%%%%%%%%
%%%  UNUSUAL PACKAGES        %%%%
%%%  Uncomment as necessary. %%%%
%%%%%%%%%%%%%%%%%%%%%%%%%%%%%%%%%

%% MATH AND PHYSICS SYMBOLS
%% ------------------------
%\usepackage{slashed}       % \slashed{k}
%\usepackage{mathrsfs}      % Weinberg-esque letters
%\usepackage{youngtab}	    % Young Tableaux
%\usepackage{pifont}        % check marks
%\usepackage{bbm}           % \mathbbm{1} incomp. w/ XeLaTeX 
%\usepackage[normalem]{ulem} % for \sout


%% CONTENT FORMAT AND DESIGN (below for general formatting)
%% --------------------------------------------------------
\usepackage{lipsum}        % block of text (formatting test)
%\usepackage{color}         % \color{...}, colored text
%\usepackage{framed}        % boxed remarks
%\usepackage{subcaption}    % subfigures; subfig depreciated
%\usepackage{paralist}      % compactitem
%\usepackage{appendix}      % subappendices
%\usepackage{cite}          % group cites (conflict: collref)
%\usepackage{tocloft}       % Table of Contents	
\usepackage{wrapfig}		% text wrap

%% TABLES IN LaTeX
%% ---------------
%\usepackage{booktabs}      % professional tables
%\usepackage{nicefrac}      % fractions in tables,
\usepackage{multirow}      % multirow elements in a table
%\usepackage{arydshln} 	    % dashed lines in arrays

%% Other Packages and Notes
%% ------------------------
%\usepackage[font=small]{caption} % caption font is small

\usepackage[hang,flushmargin]{footmisc} % no indent



%%%%%%%%%%%%%%%%%%%%%%%%%%%%%%%%%%%%%%%%%%%%%%%
%%%  PAGE FORMATTING and (RE)NEW COMMANDS  %%%%
%%%%%%%%%%%%%%%%%%%%%%%%%%%%%%%%%%%%%%%%%%%%%%%

\usepackage[margin=2cm]{geometry}   % reasonable margins

\graphicspath{{figures/}}	        % set directory for figures

% for capitalized things
\newcommand{\acro}[1]{\textsc{\MakeLowercase{#1}}}    

\numberwithin{equation}{section}    % set equation numbering
\renewcommand{\tilde}{\widetilde}   % tilde over characters
\renewcommand{\vec}[1]{\mathbf{#1}} % vectors are boldface

\newcommand{\dbar}{d\mkern-6mu\mathchar'26}    % for d/2pi
\newcommand{\ket}[1]{\left|#1\right\rangle}    % <#1|
\newcommand{\bra}[1]{\left\langle#1\right|}    % |#1>
\newcommand{\Xmark}{\text{\sffamily X}}        % cross out

% Change list spacing (instead of package paralist)
% from: http://en.wikibooks.org/wiki/LaTeX/List_Structures#Line_spacing
\let\oldenumerate\enumerate
\renewcommand{\enumerate}{
  \oldenumerate
  \setlength{\itemsep}{1pt}
  \setlength{\parskip}{0pt}
  \setlength{\parsep}{0pt}
}

\let\olditemize\itemize
\renewcommand{\itemize}{
  \olditemize
  \setlength{\itemsep}{1pt}
  \setlength{\parskip}{0pt}
  \setlength{\parsep}{0pt}
}


% Commands for temporary comments
\newcommand{\comment}[2]{\textcolor{red}{[\textbf{#1} #2]}}
\newcommand{\flip}[1]{{\color{red} [\textbf{Flip}: {#1}]}}
\newcommand{\email}[1]{\texttt{\href{mailto:#1}{#1}}}

\newenvironment{institutions}[1][2em]{\begin{list}{}{\setlength\leftmargin{#1}\setlength\rightmargin{#1}}\item[]}{\end{list}}


\usepackage{fancyhdr}		% to put preprint number



% Commands for listings package
%\usepackage{listings}      % \begin{lstlisting}, for code
%
% \lstset{basicstyle=\ttfamily\footnotesize,breaklines=true}
%    sets style to small true-type


%%%%%%%%%%%%%%%%%%%%%%%%%%%%%%%%%%%%%%%%%%%%%%
%%%  TIKZ COMMANDS FOR EXTERNAL DIAGRAMS  %%%%
%%%  requires -shell-escape               %%%%
%%%  in texpad 1.7: prefs > shell esc sec %%%%
%%%%%%%%%%%%%%%%%%%%%%%%%%%%%%%%%%%%%%%%%%%%%%

%% This is for exporting tikz figures as into a ./tikz/ subfolder.
%% It is useful if you want pdf versions of the tikz diagrams or
%% if you need to speed up compilation of a large document with
%% many tikz diagrams.

%\write18{} % Careful with this!
%\usetikzlibrary{external}
%\tikzexternalize[prefix=tikz/] % folder for external pdfs


%%%%%%%%%%%%%%%%%%%
%%%  HYPERREF  %%%%
%%%%%%%%%%%%%%%%%%%

%% This package has to be at the end; can lead to conflicts
\usepackage{microtype}
\usepackage[
	colorlinks=true,
	citecolor=black,
	linkcolor=black,
	urlcolor=green!50!black,
	hypertexnames=false]{hyperref}



%%%%%%%%%%%%%%%%%%%%%
%%%  TITLE DATA  %%%%
%%%%%%%%%%%%%%%%%%%%%

%%% PREPRINT NUMBER USING fancyhdr
%%% Don't forget to set \thispagestyle{firststyle}
%%% ----------------------------------------------
%\renewcommand{\headrulewidth}{0pt} % no separator
%\fancypagestyle{firststyle}{
%\rhead{\footnotesize \texttt{UCI-TR-2016-XX}}}



\begin{document}

%\thispagestyle{empty}
%\thispagestyle{firststyle} %% to include preprint

\begin{center}

    {\Large \textsc{P231:} \textbf{Methods of Theoretical Physics} (Fall 2020)}
    
\end{center}

\vskip .4cm

\noindent
\begin{tabular*}{\textwidth}{rlcrl}
\textsc{Lec:}& Flip Tanedo (\email{flip.tanedo@ucr.edu})
&
\hspace{1cm}
&
\textsc{Meet:} & 
MWF 10 -- 10:50am Chung 139
\\
\textsc{TA:}& TBD
&
\hspace{1cm}
&
\textsc{Dis:} & 
M 3 -- 3:50pm MSE 113 (TBD)
\end{tabular*}

\vspace{1em}
\noindent We do \emph{not} anticipate using the discussion time regularly.

\subsection*{Critical Information}
\textsc{web page:} \url{https://sites.google.com/ucr.edu/p231/}

% \vspace{.5em}
\noindent Lecture notes, homework and our course calendar will be linked from the course web page. 

\vspace{.5em}
\noindent Graduate students are \emph{not} required to take this course. You are welcome to drop the course and use this time more productively\footnote{Suggestions include: going for a walk to stretch, working on other assignments, reading new academics papers, walking through the botanical garden, talking to potential advisers, $\cdots$}. 

% \vspace{.5em}
% \noindent This course is offered remotely. Expect 3--5 short (10 min) recorded min-lectures each week to supplement a set of course notes and assigned readings. The focus of this course is solving the problems, \emph{not} the lectures. 


\subsection*{Course Description}

This is a crash course in mathematical methods for physics and the technical communication skills that will be necessary for your scientific career. The topics are selected to ensure students are prepared for the first-year graduate curriculum at \textsc{ucr}. Our main goal is to solve differential equations using Green's functions. To do this, we will develop and use techniques from linear algebra and complex analysis. If time permits, we may explore other topics toward the end of the course such as statistical methods for physics and astronomy.
%
This is not a mathematics course, it is \emph{boot camp} for physicists. 

\subsection*{Evaluation}

\begin{itemize}
\item \textbf{Short Assignments} (20\%): Every \emph{two} weeks you will receive a short assignment that is due \emph{Wednesday}, two days after it is assigned. Students may be asked to present solutions during the lecture. The goal of these assignments is to review basic ideas from lecture and identify any confusion \emph{that you should ask in class}. 

\item \textbf{Explainer Videos} (40\%): Every \emph{two} weeks you will be assigned a \emph{long assignment}. You are strongly encouraged to do all of the problems, but you will be graded on one randomly assigned problem where you will record a video (up to 10 minutes long) explaining how to solve the problem. You will upload a pdf file of your solution. These videos will be shared with the class. 

\item \textbf{Peer Review} (20\%): Every \emph{two} weeks you will review five videos by your peers. You will grade them based on a rubric and you will provide constructive feedback. 

\item \textbf{Surveys} (10\%): Each week you will complete a short survey with review questions and requests for metacognitive\footnote{\emph{metacognition}: awareness and understanding of one's own thought processes.} feedback.

\item \textbf{Essay} (10\%): You will have a one-time written assignment to prepare a `how to' guide on solving the harmonic oscillator using Green's functions. 

\item No exams. We will not use the university exam period.
\end{itemize}

I expect you to \emph{work together} and to abide by the \href{http://conduct.ucr.edu/policies/academicintegrity.html}{UCR academic integrity policies}. You are free to use whatever resources you have available; please cite sources appropriately. When in doubt, cite.

\subsection*{Course Objectives}

The contents of this course build a mathematical foundation that is at the core of graduate-level physics and astronomy. The topics are chosen to provide a foundational understanding of the mathematical methods needed in the first year graduate curriculum.

The course methodology is designed to build soft skills necessary to succeed in academia. Being able to effectively communicate one's technical work (or even one's technical confusion) is a key skill for the rest of your scientific careers, academic or otherwise.
%
This is an unusual time to be starting graduate school. Our remote learning goals are to (1) use this as an opportunity to engage with ideas in a way that is \emph{more} aligned with the way you will learn as a Ph.D student, and (2) build community within your graduate cohort. 


\subsection*{Textbook}

There is no required textbook. Course notes and a list of suggested references are posted online, including low-cost Dover edition books and no-cost digital books through the \textsc{ucr} library. You are strongly encouraged to have \emph{some} mathematical physics reference available.

\subsection*{Technology requirements}

For this course, you will need to be able to record 5--10 minute videos of yourself explaining the solutions to homework problems. There are many ways to do this, check out the UCR Keep Learning website\footnote{\url{https://keeplearning.ucr.edu/recording-video-presentationsperformances}} for suggestions. Your videos do not need to be polished: you need to be effective, not flashy. You are encouraged to arrange for your recording to show your face while talking if possible; this will help us build familiarity with one another. At the very minimum, your videos must be narrated in your own voice. 

In the second half of the course you will prepare a short written document explaining how to solve for the Green's function of a harmonic oscillator. You are strongly encouraged to use \LaTeX. 


\section*{Topics}

The main theme of the course will be understanding how to solve the partial differential equations that pop up in physics using Green's functions. The rough number of weeks is an estimate.

\begin{enumerate}
	\item \textbf{Dimensional analysis}. [1 week] How do you tell a physicist from a mathematician?
	\item \textbf{Differential equations}. [2 weeks] Are differential equations just linear algebra?
	\item \textbf{Complex Analysis}. [2 week] How do I integrate around poles?
	\item \textbf{Green's functions}. [3 weeks] How do I solve differential equations? 
	\item \textbf{Variational principles}. [1 weeks] Where did these equations come from? 
	\item \textbf{Special Topics}. [1 weeks]  Special topics to be decided. Possibilities include: probability and statistics (how do you know when you've discovered something?), statistical learning (what is machine learning?), differential geometry (what is a magnetic monopole?).
\end{enumerate}

\section*{Learning Objectives}

By the end of this course, you are expected to attain the following learning outcomes:
\begin{enumerate}
	\item Use dimensional analysis to determine scaling relations and estimate the solutions to problems.
	\item Apply the tools of linear algebra to differential equations by treating differential operators as infinite dimensional linear transformations.
	\item Use Green's functions to solve inhomogeneous linear differential equations.
	\item Solve for Green's functions in multiple ways, including an eigenfunction expansion and as an integral transform.
	\item Solve basic complex coutour integrals that show up in the harmonic oscillator and identify the physical consequences of analyticity (e.g.~ dispersion)
	\item Apply Green's function methods to basic problems in electrodynamics
\end{enumerate}
Additionally, following soft skills will be emphasized:
\begin{enumerate}
	\item Thinking critically about the process of how you solve problems.
	\item Thinking critically about how your peers solve problems.
	\item How to present technical ideas to a peer audience.
	\item How to ask and answer questions in an academic setting.
\end{enumerate}

\section*{Teaching Team}

\noindent \textsc{Instructor}: Prof.~Flip Tanedo (he/him) is a particle physicist who specializes in theories of dark matter. He enjoys science fiction on screen (\emph{Star Trek}) and as short stories (recent favorites: N.K.~Jemisen and Ted Chiang). As graduate students you are invited to address your faculty by their first names---you are a young colleague, no longer just a `student.' 

\vspace{.5em}
\noindent \textsc{Teaching Assistant}: To be determined.

\section*{General Advice}

I strongly encourage you to ask questions and engage with one another, for example through our Slack workspace. There are two questions that you can \emph{always} ask:
\begin{enumerate}
	\item ``\emph{Is it obvious that...?}'' This means: I don't know if I fully understand something. Maybe I'm looking at it the wrong way, what is the best way to see that this is true? 
	\item ``\emph{Why are we doing this?}'' You may understand the details, but have lost track of the big picture. What is the main point of this section?
\end{enumerate}
These are good ways to clarify what we're doing without worrying about ``appearing dumb'' for asking them.

\section*{Course Expectations}

\begin{itemize}
	\item This is an elective course meant to provide the mathematical preparation for our department's first year graduate curriculum. If you already feel comfortable with the material in the course, you are invited to drop the class.
	\item Attendance is not mandatory but is strongly encouraged. You will get more out of this class from asking questions and engaging during lecture.
	\item The minimum workload has been chosen to be modest. You are expected to complete this minimum on time.
	\item Short assignments are \emph{safe places to fail}. This is where you try applying an idea in lecture. You will either turn in something with reasonable confidence, or you will turn in something that you are unsure of \emph{and then are expected to ask for clarification in lecture}. 
	\item Long assignments are your chance to demonstrate mastery. Creating an explainer video and having it peer reviewed is a small version of how ideas are shared in academia. 
	\item The surveys are brief and informal checks of the week's material and are your best way to give me candid feedback.
	\item Extra credit problems are not graded, but they are there to provide a framework for those with further interest in the material. If you see a problem related to your [potential] research direction, you should really attempt it.
	\item 	\textbf{What to do if you're stuck on a problem}. The suggested course of action is:
	\begin{enumerate}
		\item Ask your peers.
		\item Ask the TA.
		\item Ask in class.
		\item If there is a problem that is likely related to an error in the homework, you can reach out to Flip using \textsc{[231]} in the subject of the email.
	\end{enumerate}
	\item \textbf{Email communications}: Unfortunately, due to the sheer volume of email professors receive, it may take me up to a week to respond to messages. Please know that even if I do not respond right away, I do read these within a day.
	\item \textbf{Digital documents}: Assignments are submitted online as \acro{PDF} files. The suggested methods are: (1) using \LaTeX to type up your assignments\footnote{One easy way to include graphics is to write them out on paper and insert a photo into your document.}, (2) uploading a photo\footnote{You may want to use a scanning app to help with processing into a \acro{PDF}}, or (3) using the photocopier in the Barkas lounge (3rd floor, Physics Building) to email yourself a scanned copy of your written work. We do this so that you get to keep a copy of your work (it may help for subsequent assignments).
\end{itemize}


\section*{Our Values \& Learning Philosophy}

This is an unusual year to be a first-year graduate student. Our goal with this course is to adapt in a meaningful way to make this experience valuable. Our guiding principles are:

\begin{itemize}
	\item \textbf{Your time and attention are precious.} Remote learning can quickly cause Zoom fatigue. We want to allow as much flexibility in your lives.

	\item \textbf{Your words are more important than mine.} Rather than traditional lecture, it's better for us to find more meaningful ways to engage with the material and each other. This will be a flipped classroom where we focus on you solving problems rather than hearing me tell you about solving problems.

	\item \textbf{Your cohort is valuable.} Your grad school colleagues are your most valuable allies. Find effective ways to collaborate with one another. \emph{Respect one another}, both as human beings and as allies on a shared academic journey\footnote{Nobody will better understand the challenges of being a graduate student in this place and at this time than your classmates. When you graduate and launch your scientific careers, it will be you competing for positions against similar graduates from other departments across the world. Your classmates now are your best allies to push yourself to be better and to support one another through the challenges ahead.}.

	\item \textbf{This is a support class.} There is no comprehensive exam test for this course. The main purpose of this course is to make sure you have the mathematical tools to succeed in your grad classes this year. This class is not trying to ``weed out'' any students or place undue burden on your attention. 

	\item \textbf{Communication is key.} As young scientists, you will be judged as much on your ability to communicate your science verbally as your ability to `do' your science\footnote{I firmly believe that being able to communicate is part of the definition of `doing' science.}. You will present your homework as videos because this trains your for every oral exam, conference talk, and interview that you have ahead of you.
	
	\item \textbf{You get what you put in.} You must work through problems to get anything out of this course. You don't have to do many and it doesn't have to take much time, but you will get nothing from just `watching' this course.
\end{itemize}

\section*{COVID Policies}

Please refer to the UCR Campus Return website\footnote{\url{https://campusreturn.ucr.edu}} for university-wide policies regarding \acro{COVID-19}. If you feel ill and suspect that you may have contracted \acro{COVID-19}, be sure that you have filled out the Daily Wellness check and follow the procedures for case reporting\footnote{\url{https://campusreturn.ucr.edu/case-reporting-and-investigation}}. The lecture notes and assignment submission portals for this course are available online and we expect you to be able to follow remotely. Please contact the professor if you feel you need additional accommodation while self-isolating for an \acro{COVID-19} case.

We are reminded by the university that events may occur that would mandate a temporary return to remote teaching. One example for our class would be if the instructor tests positive for \acro{COVID}. In the event that we return to remote teaching, the lecture will be offered over Zoom with recordings made available. 


\section*{Policies}

\begin{itemize}
	\item \textbf{Course Load}. By \textsc{ucr} Senate Regulation 760, one unit of course credit corresponds to 3 hours of course work per week (including time in class). This is a 4 unit course and so you are expected to spend up to 12 hours a week. Most of this time will be spent working on problems, preparing your videos, and interacting with others. I anticipate that one can meet the learning objectives without using the full 12 hours, but let me know if you are spending significantly more than 12 hours on this course.
	\item \textbf{Equity and Inclusion}. We are committed to creating an inclusive learning space where we respect one another regardless of race/ethnicity, gender identity or expression, sexual orientation, socioeconomic status, age, disabilities, religion, regional/national background, veteran status, citizenship status, and other diverse identities that we each bring to class.
	\item \textbf{No bullying}. This course (and grad school in general) requires students to share work with one another. We will treat each other with respect in our constructive criticism and we will not share each others' materials outside our course without their explicit and written permission. Do not be a troll or bully anyone in this course; we are each offering some vulnerability to support this learning environment.  \emph{Seriously, don't be an asshole.}
	% \item \textbf{Communication}. Please use Slack as the primary way of communicating for physics questions and administrative questions. If there are administrative issues that are specific to you that you do not want to share, you can email the professor. Please use \texttt{[P231]} in the subject of the email and anticipate a 2-day turnaround time. Major course announcements will be sent through \texttt{iLearn}, which should forward to your email.
	% \item \textbf{What to do if you're stuck on a problem}. The suggested course of action is:
	% \begin{enumerate}
	% 	\item Ask in the appropriate Slack channel. 
	% 	\item If there isn't an adequate response, you can reach out to the \textsc{ta} by email, please use \texttt{[P231]} in the subject of the email.
	% 	\item If there is still confusion (i.e.~there's likely an error on the homework---sorry) you can reach out to the professor by email, please use \texttt{[P231]} in the subject of the email.
	% \end{enumerate}
	% \item \textbf{Attendance}. Most of this course is asynchronous. However, your participation in the Slack workspace is important; please make it a habit to review the workspace regularly. Individual one-on-one interviews will be arranged at a time that is as mutually convenient as possible.
	\item \textbf{Late homework}. Due to the peer review aspect of this course, late homework will not be accepted. This is stricter than my past policies, but in exchange we are reducing the amount of work required for submission.
	\item \textbf{Academic Integrity}. All students are expected to abide by the highest standards of academic integrity\footnote{\url{https://conduct.ucr.edu/policies/academic-integrity-policies-and-procedures}}. Academic misconduct (cheating) will be reported to the \textsc{ucr} Student Conduct \& Academic Integrity Programs and will be penalized to the fullest amount. A brief summary:
	\begin{itemize}
		\item You are encouraged to collaborate with others on homework and presentations. 
		\item You are expected to write your solutions based on your own understanding.
		\item You are allowed you use \emph{any} references outside of the assigned course materials. You are expected to cite these sources in your submitted work or presentations.
		\item Always cite. When in doubt, ask ahead of time.
	\end{itemize}

\end{itemize}



\section*{Inclusive Accommodation, Support}
\begin{itemize}
	\item Students who need any accommodations that require my attention should contact me in the first week of class. Students with permanent or temporary disabilities should be sure to make accommodations with the Student Disability Resource Center\footnote{\url{https://sdrc.ucr.edu/}}.

	\item We are committed to an inclusive classroom where our views may be challenged, but where we will always respect each other's dignity and humanity. We each have a responsibility to hold ourselves and one another (including faculty) accountable for maintaining this standard. In the case of any incidents in the classroom, we will (1) find a respectful resolution together, or if this is not possible (2) discuss with the necessary parties outside of the class, or if neither is feasible, (3) reach out to either {Help at \textsc{ucr}}\footnote{\url{https://help.ucr.edu}} and/or the {Office of the Ombuds}\footnote{\url{https://ombuds.ucr.edu}}. Please know that all University of California staff and faculty are designated Title IX responsible employees which means we are required to report any instances of sexual violence or sexual harassment to our Title IX office; if you are looking for a confidential source of support, please reach out to the \textsc{ucr care} office\footnote{\url{https://care.ucr.edu/}}.

	\item If at any time in this course or in your time at \textsc{ucr} you should feel comfortable reaching out to Counseling \& Psychological Services\footnote{\url{https://counseling.ucr.edu/}} if you are feeling distress or anxiety. This is a commonly used resource for graduate students.

	\item Should you need modifications or adjustments to your course requirements because of documented pregnancy-related or childbirth-related issues, please contact me as soon as possible to discuss your options. Generally, modifications will be made where medically necessary and similar in scope to accommodations based on temporary disability.  Learn more about the rights of pregnant and parenting students by consulting the Office of Diversity, Equity, and Inclusion\footnote{\url{https://diversity.ucr.edu}}.

	\item It is the policy of the University to excuse absences of students that result from religious observances and to provide for the rescheduling of examinations and additional required classwork that may fall on religious holidays without penalty. It is the responsibility of \emph{the student} to make alternate arrangements with the instructor at least one week prior to the actual date of the religious holiday.
\end{itemize}


\end{document}